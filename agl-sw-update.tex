% v2-acmsmall-sample.tex, dated March 6 2012
% This is a sample file for ACM small trim journals
%
% Compilation using 'acmsmall.cls' - version 1.3 (March 2012), Aptara Inc.
% (c) 2010 Association for Computing Machinery (ACM)
%
% Questions/Suggestions/Feedback should be addressed to => "acmtexsupport@aptaracorp.com".
% Users can also go through the FAQs available on the journal's submission webpage.
%
% Steps to compile: latex, bibtex, latex latex
%
% For tracking purposes => this is v1.3 - March 2012

\documentclass[prodmode]{kgwpstd} % Aptara syntax

% Package to generate and customize Algorithm as per ACM style
\usepackage[ruled]{algorithm2e}
\renewcommand{\algorithmcfname}{ALGORITHM}
\SetAlFnt{\small}
\SetAlCapFnt{\small}
\SetAlCapNameFnt{\small}
\SetAlCapHSkip{0pt}
\IncMargin{-\parindent}

\copyright{Copyright 2016 Konsulko Group}
\license{CC BY-SA 3.0 US, \url{https://creativecommons.org/licenses/by-sa/3.0/us/}}

% Document starts
\begin{document}

% Page heads
\markboth{Konsulko Group}{Device-Side Software Update Strategies for Automotive Grade Linux}

% Title portion
\title{Device-Side Software Update Strategies for Automotive Grade Linux}
\author{Matt Porter}
\affil{Konsulko Group}
% NOTE! Affiliations placed here should be for the institution where the
%       BULK of the research was done. If the author has gone to a new
%       institution, before publication, the (above) affiliation should NOT be changed.
%       The authors 'current' address may be given in the "Author's addresses:" block (below).
%       So for example, Mr. Abdelzaher, the bulk of the research was done at UIUC, and he is
%       currently affiliated with NASA.

%\begin{abstract}
%	INSERT ABSTRACT HERE
%\end{abstract}

%\category{A}{Automotive SOTA}{SOTA}

%\terms{AGL, Software, Update, Security}

%\keywords{software update, atomic update,
%trusted execution environment, security}

%\acmformat{Matt Porter, Device-Side Software Update Strategies for Automotive Grade Linux}
% At a minimum you need to supply the author names, year and a title.
% IMPORTANT:
% Full first names whenever they are known, surname last, followed by a period.
% In the case of two authors, 'and' is placed between them.
% In the case of three or more authors, the serial comma is used, that is, all author names
% except the last one but including the penultimate author's name are followed by a comma,
% and then 'and' is placed before the final author's name.
% If only first and middle initials are known, then each initial
% is followed by a period and they are separated by a space.
% The remaining information (journal title, volume, article number, date, etc.) is 'auto-generated'.

\begin{bottomstuff}
This work is supported by Advanced Telematic Systems GmbH
\par
\end{bottomstuff}

\maketitle

\section{Overview}

This whitepaper explores the area of software update strategies for devices
running Automotive Grade Linux. The starting point is understanding several key
use cases for updating software in an
AGL\footnote{\url{https://www.automotivelinux.org/agl-specification}} system.
Several open source device-side software update mechanisms are compared with a
focus on their ability to meet the stated use cases.  Finally, recommendations
are made for an approach that can be implemented for inclusion in AGL.


\section{Use Cases and Requirements}

The topic of software update on any computing device is very broad and
can only be examined properly by narrowing the scope of the system,
operating conditions, and the policies established for executing a
software update. The following sections describe the use cases
considered when evaluating the various mechanisms that could be employed
in an AGL software update strategy.

\subsection{System Description}

AGL systems include both IVI\footnote{In-Vehicle Infotainment} and
ADAS\footnote{Advanced Driver Assistance System} devices that run Linux on at
least one processor in the system. The complete IVI or ADAS system is not
necessarily limited to a single processing node, but is normally part of a
larger network that includes a number of ECU\footnote{Electronic Control Unit}
nodes throughout the automobile. Due to this distributed architecture it is
often necessary to update software on the ECUs as well as the node running AGL.
The device running AGL is generally the node containing networking or other
external communication capabilities to handle input of a software update image.
As such, it will normally be the focal point for software update in the entire
automobile providing update services for itself and multiple ECUs. For the AGL
node, an update mechanism must support update of all software loadable
bootloaders, the kernel and supporting configuration data, and all Linux
filesystems as the system design may or may not require updating any and all of
these software artifacts. Finally, AGL’s build system is based on OpenEmbedded
so any technology must be able to integrate into a well-behaved OpenEmbedded
build environment.

\subsection{Operating Conditions}

An automobile product has several operating conditions that are both unique and
also share similarities with other devices. First, and foremost, the automobile
is a mobile device, which requires the ability to be updated at any location
with minimal impact to the owner. The automobile industry has historically
relied on costly on-site visits to service facilities to update software,
however, with the increasing complexity and amount of software in an automobile
this approach does not scale. For this reason, support of OTA\footnote{Over The
Air} software updates via wireless networking has become a fundamental use
case.

The automobile industry is subject to numerous safety regulations. As such,
automobile software updates must also be conducted in a manner in which safety
is not compromised. Failure is common and expected during software updates with
causes that include image corruption during installation, corrupted
image/filesystem on storage medium, failure to receive update image, and power
failure during update. It is critical that a software update strategy support
deployment of updates such that if failure occurs, a previous version of the
software can be deployed such that the automobile is functional and safe to
operate until a new update can be deployed. In order to minimize the
possibility of failure, updates must be deployed in an atomic manner,
guaranteeing integrity of the software update once it is deployed on the
device. In addition, the cost nature of OTA delivery methods such as 4G
networks requires QoS to be implemented to control the costs of deploying large
software updates to a fleet of vehicle. A software update may need to be
delivered at specific time of day and at a specific bandwidth rate
corresponding to the region the automobile is sold or located within. Finally,
the update mechanism must be able to verify that the software has not been
tampered with before installation and must cooperate with a system-level chain
of trust during the boot process that verifies images starting a power-on
through to application lifecycle. This is commonly handled by executing in a
trusted environment that leverages hardware features such as a Trusted Platform
Module (TPM) and Trusted Execution Environment (TEE).  Popular TEEs include ARM
TrustZone and Intel TXT.

\subsection{Policies}

A software update mechanism needs the ability to support a flexible set of OEM
software update policies for the automobile. Each OEM has a different approach
to the rate of updates, quantity, and size of updates. An OEM with a higher
rate of updates will require an update mechanism that’s optimized not only for
the baseline requirements but also for speed. Some OEMs may provide value-add
features that are unlocked by a recurring fee. These value-add features will
require the ability to enable or disable a specific feature based on a fee.
Regulations requiring a recall may require the OEM to rollback an update due to
a recall notice. Finally, the policy of update deployment timing will be
defined by the OEM to control operator experience (e.g. only at “key-off” and
with operator acknowledgement) as well as meet any safety requirements.

\subsection{Summary}

The following requirements are derived from the use cases and are listed in
priority order:
% enumerate
\begin{enumerate}
	\item Atomic software release update
	\item On failure, deploy previous working bootloader, kernel and
		configuration, and filesystems on AGL device
	\item Update of bootloader, kernel and configuration
		data, and filesystems on AGL device
	\item Support for OpenEmbedded-based builds
	\item Support for updating both the AGL device and any ECU devices
	\item Flexible delivery of software image(s) with QoS
		\footnote{Quality of Service} controls and supporting arbitrary
		interfaces (WiFi, 4G, USB, etc.)
	\item Support for signing of images and verification of images on
		installation
	\item Support trusted boot and execution of software update in a
		trusted application environment leveraging the platform’s
		hardware TPM and/or TEE features.
	\item Enable/disable a specific feature and apply/rollback system
		updates incrementally rather than through a complete OS update
		that replaces the filesystem.
\end{enumerate}

\section{Open Source Software Update Tools}

Historically, software update on embedded Linux systems has been
implemented in a “one-off” manner by companies shipping products based
on Linux. Typically, the details of how an update is performed has been
very specific to the system and hardware design as well as the amount of
storage space available to implement a software update mechanism. Over
time, these ad hoc update methods have matured to where requirements
have begun to show some commonality allowing the problem of software
update to be solved in a general manner. The following projects provide
implementations of common requirements for embedded Linux software
update.

\subsection{SWUpdate}

SWUpdate\footnote{\url{https://sbabic.github.io/swupdate/swupdate.html}} is an
extensible software update framework that supports atomic update of arbitrary
software images in a Linux system. SWUpdate defines a standard compound image
format based on cpio which contains a header, software description in XML, and
any number of software update sub-images. It allows for extensible image
parsers to support new software image types as well as extensible handlers to
support new protocols or installers for specific storage peripherals and
layouts.  The XML-based software description provides a manifest of all the
software sub-images contained in the compound image. It defines attributes
corresponding to the version of a software release and the name and type of
each sub-image contained within that software release.  For example, a software
release for a typical Linux system may have sub-images for the zImage, dtb,
root filesystem, and apps filesystem.  The compound image and each sub-image
can be hashed with SHA256 and signed with an AES256 key for use in verification
before installation.

SWUpdate can operate in either a single copy or dual copy update scheme.
Single copy updates require updates to be managed by the bootloader booting
into a kernel and initrd that then runs the SWUpdate tool to install a new set
of software images. This approach conserves space but does not allow for
fallback to a known good software image on failure.  Rather, it requires
reinstall of a previously failed image or download of a new software after
reboot from a failed install. This approach is improved upon by using a typical
dual copy approach which involves keeping a known good recovery copy of the
software in a second partition. SWUpdate supports a dual copy software update
strategy by use of Software Collections. A Software Collection is simply two
named alternate copies of software images with attributes defined to describe
what the target installation location is for each copy. This allows SWUpdate to
be called to deploy to either storage partition by specifying which Software
Collection to deploy to the storage device.

There’s no explicit support for integrating SWUpdate into a TEE but there’s no
reason it could not be ported to such an environment.  SWUpdate provides a
reference mongoose webserver for use as a target-based software update user
interface. However, it also supports an API for arbitrary software to
communicate with it in order to install images and receive status on
installation status. It is possible to integrate the RVI SOTA client with
SWUpdate using this API to perform the installation. For compound image
generation, SWUpdate provides a
meta-swupdate\footnote{\url{https://github.com/sbabic/meta-swupdate/}} layer
which supports generation of signed images in OpenEmbedded. The project has a
small community around it and is also included in the
buildroot\footnote{\url{https://buildroot.org/}} project.

\subsection{Mender}

Mender\footnote{\url{https://mender.io/}} is a dual copy oriented framework
that supports the end-to-end management of root filesystem updates and
rollback. It supports a OTA delivery server and provides a target side software
update tool to manage deployment of updates. Mender specifically supports only
U-Boot as the bootloader and implements support for rollback to the second copy
of an OS using U-Boot’s \emph{bootlimit} feature.  It does not support update of the
bootloader itself nor does it support incremental updates.

Mender is written in Go and provides a \emph{meta-mender} layer for OpenEmbedded to
support building embedded systems including Mender support. There’s no explicit
support for integrating Mender into a TEE but there’s no reason it could not be
ported to such an environment.

Mender does not appear to have any community adoption as evidenced by list
posts and commits only originating from Mender Software itself.

\subsection{Resin}

Resin\footnote{\url{https://resin.io/}} is a container-based frame for
delivering rolling updates to embedded Linux systems. It is a client/server
system where the server has support for building packages and containers with
the package content. In addition, the server has a docker registry for the
containerized applications. On the target, a supervisor application runs in a
container providing monitoring of the device as well as handling rollout of new
docker containers with applications. This can all be managed from the cloud
infrastructure maintained by resin.io.

Resin has no provision for managing updates to the core operating system
binaries. It is completely focused on an immutable and non-updated base OS and
fluidly updated applications within containers. These containers are stateless
except that a \emph{/data} directory is provided that will persist per device
between updates. Currently, support for binary deltas of Docker container
updates is a feature that is in beta. Resin makes use of TLS for the
client/server connection using RSA for key exchange and AES256 for data
encryption. There’s no explicit support for integrating Resin OS into a TEE but
there’s no reason it could not be ported to such an environment. A \emph{meta-resin}
layer for OpenEmbedded is provided to support building the base Resin OS for
deployment on target devices.

Resin.io started as a commercial service only and then moved to open source
their framework in late 2015. As Resin OS has support for basic I/O on several
popular community boards (Raspberry Pi, Edison, BeagleBone) it seems to have
gained some community support as evidenced by commits from those outside of
Resin itself.

\subsection{swupd}

swupd\footnote{\url{https://github.com/clearlinux/swupd-client}} is a
revisioned software update mechanism which was designed to meet the
requirements of the ClearLinux\footnote{\url{https://clearlinux.org/}}
distribution project from Intel. It is intended for use in Linux systems that
update software in small increments at a rapid pace. Swupd does not use
packages as a unit of update like traditional Linux distributions. Instead,
swupd defines bundles which are a composed of a set of specific package
versions. A bundle is the smallest installable component in an swupd system.
These bundles are combined into a OS release that carries a single version
number. By default, swupd demands that the Linux distribution be designed to
operated as a stateless Linux OS. That is, an unpopulated /etc/ directory
results in a bootable system that is factory reset.

Swupd provides client and server side tools to manage updates. The swupd-client
support adding and removing specific bundles or updating to a new OS release
atomically. In addition, the client supports checking the update server for a
new OS release. There’s no explicit support for integrating swupd-client into a
TEE but there’s no reason it could not be ported to such an environment. The
swupd-server provides tools to create bundles, create OS releases, and host
them for consumption by the client. The meta-swupd[14] layer integrates the
swupd-server tools for generation of bundles and OS releases for OpenEmbedded
as well as the swupd-client tools to support software updates on the target
filesystem.

An OS release is defined by a “Manifest of manifests” which lists each bundle
that’s part of a release. Each bundle has a manifest which lists each change to
a file, directory, symlink, etc. Swupd makes use of bsdiff (binary diff
utility) to minimize the size of an OS update. Each manifest and change is
identified using HMAC-256 to support verification of the OS release content.
When an update is requested, the client downloads all manifests, then uses
those to determine the packs of updates to download, and then applies each
change in sequence. Post update scripts can be triggered using the systemd
update-triggers.target.

Swupd was created for the ClearLinux distribution and also adopted by the Ostro
OS\footnote{\url{https://ostroproject.org/}} distribution, both of which are
Intel project. The project mailing lists are dominated by Intel employees and
it appears there’s no adoption of swupd outside of these Intel projects.

\subsection{OSTree}

OSTree is similar in scope to swupd, providing support for atomic upgrade of
Linux filesystems. A server composes content to be used on a client system
where OSTree manages deployment of the content. By default, it defines two
persistent directories across updates, \emph{/etc} and \emph{/var}, rather than operating
as a stateless system like many systems. OSTree is best described as a
configuration management tool for filesystem binaries. It allows for multiple
releases to be stored as deltas of files, generated using bsdiff on the server
or build system.  OSTree internally follows the design of git using tree
objects to track the history of filesystems including both content and
metadata. It uses SHA-256 to hash changes in the tracked filesystems and
GPG\footnote{\url{https://www.gnupg.org/}} is used to sign and verify
commits/releases.

An administrative tool is provided on top of this version control system to
management deployment of filesystems. \emph{ostree admin upgrade} will update a
deployment tree to the latest release on the update server and prepare the
system to boot that release on the next boot. Likewise, \emph{ostree admin deploy}
and \emph{ostree admin undeploy} can deploy or rollback a specific commit or version
for the next boot. OSTree can also manage bootloader configuration files that
conform to the boot loader
specification\footnote{\url{https://www.freedesktop.org/wiki/Specifications/BootLoaderSpec/}}.
The Gnome Continuous project maintains a base Gnome OS that conforms to the
distribution needs (/usr
Merge\footnote{\url{https://www.freedesktop.org/wiki/Software/systemd/TheCaseForTheUsrMerge/}})
and patches to support OSTree compliant builds for OpenEmbedded have been
previously
submitted\footnote{\url{http://lists.openembedded.org/pipermail/openembedded-core/2013-August/083595.html}}.There’s
no explicit support for integrating the \emph{ostree admin} tool into a TEE but
there’s no reason it could not be ported to such an environment.

OSTree has been adopted by Gnome
Continuous\footnote{\url{https://wiki.gnome.org/Projects/GnomeContinuous}},
rpm-ostree/Project Atomic\footnote{\url{http://www.projectatomic.io/}}, and
flatpak\footnote{\url{https://github.com/flatpak/flatpak}}. The ostree-list is
not very high volume but shows a number of different email domains indicating
use by a number of companies and individuals including those from the Gnome
Project and Red Hat.

\subsection{Other Technologies}

There are a number of other technologies that are used in ad hoc software
update mechanisms. Many make use of a dual copy scheme as described in the
SWUpdate section. However, this is not always sufficient to manage incremental
updates. This results in sometimes layering technologies such as
overlayfs\footnote{\url{https://www.kernel.org/doc/Documentation/filesystems/overlayfs.txt}}
on top of a dual copy scheme in order to support maintenance updates without
requiring a complete filesystem update.


\section{Recommendations}

Before making the final recommendations, we compare and contrast the
pros and cons of each of the update technologies.

Both OSTree and swupd provide a robust way to manage Linux filesystem
updates when the system is working properly. Both assume that the
underlying filesystem is not corrupt and that there is no data loss.
Because of this, neither satisfies all possible failure cases on its
own. OSTree is more conservative that swupd in that it requires a reboot
on any update when deploying its newly configured filesystem. Swupd does
not seem to have wide community adoption (used only in two Intel
projects) and so the recommendation is to focus on OSTree as a component
of a software update solution for systems requiring fine-grained and
frequent updates as well as rollback of updates. Both Mender and Resin
OS seem to be primarily driven by their commercial efforts and not
widely adopted as update mechanisms by other projects. Mender is focused
on a dual-copy only strategy and Resin OS is fundamentally based on a
base OS that requires containers for all applications.

As mentioned, OSTree is not sufficient to meet all failure scenarios on
its own. In order to handle the case of a corrupted filesystem or
updating boot loaders that do not reside on a Linux filesystem it’s
necessary to couple it with another system. Certainly if everything goes
correctly, a pure OSTree system could update bootloaders on its own with
appropriate tools in the filesystem. However, nothing OSTree offers
handles the case of its own underlying filesystem become corrupt and the
need to fall back to the last known working system. This is where a dual
copy approach becomes very important in a reliable system. SWUpdate
provides a reasonable implementation of a dual copy approach and has a
number of people starting to make use of it as a simple software update
framework. It needs to be extended to support additional software
fetching APIs but the maintainer seems open to contributions.

It is our recommendation that the reference AGL software update strategy
make use of SWUpdate in a dual copy configuration and integrate OSTree
support. This allows recovery from a corrupt partition for the exception
case, but also optimizes the common case where small, incremental
updates can be quickly applied or rolled back as needed to me OEM
policy. As a part of this effort, it will be necessary to also provide a
reference port of the SWUpdate and OSTree administration tools to a TEE
to demonstrate the ability to execute the update process in a trusted
environment. With the update process executing in the TEE, the
cryptographic verification performed by the SWUpdate and OSTree update
mechanisms is now a trusted action. The software running in the TEE is
only one component of the overall attack surface of the entire software
update process. It is also critical that the client/server SOTA network
protocol and key handling processes follow best security practices or
else the actual content being delivered could be compromised. It is
recommended to support OP-TEE using an ARM QEMU target for the
proof-of-concept implementation such that anybody can download and test
the solution running on QEMU. This comprehensive approach is designed to
satisfy all the requirements set forth in this paper.

\end{document}


